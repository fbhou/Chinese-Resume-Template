\documentclass{resume}
\usepackage{zh_CN-Adobefonts_external} 
\usepackage{linespacing_fix}
\usepackage{cite}
\begin{document}
\pagenumbering{gobble}

\name{鞠泉玺}
\begin{center}
  \textbf{出生日期:}2002年10月30日
\end{center}
\contactInfo{18980568850}{qxju2020@sjtu.edu.cn}

\section{教育背景}

\datedsubsection{\textbf{上海交通大学},信息工程,\textit{本科}}{2020.09 \textasciitilde~2025.02}

\section{实习经历}

\datedsubsection{\textbf{中国电力建设集团成都勘测设计研究院有限公司数字工程分公司}}{2022.07 \textasciitilde~2022.08}
\begin{itemize}[parsep=0.5ex]
  \item 后端Java开发工程师
\end{itemize}

\datedsubsection{\textbf{光方迅视云服务项目组}}{2023.10 \textasciitilde~2024.01}
\begin{itemize}[parsep=0.5ex]
  \item React前端工程师
\end{itemize}

\section{项目经历}

\datedsubsection{\textbf{类脑神经网络损失函数可视化}}{2023.10 \textasciitilde~2024.05}
\begin{itemize}[parsep=0.5ex]
  \item 2024 届本科毕业设计
  \item 对类脑神经网络的损失函数地形进行可视化
  \item 探究超参数设置等对模型泛化性能的影响
\end{itemize}

\datedsubsection{\textbf{基于视觉分割大模型的视觉重定位技术}}{2023.06 \textasciitilde~2023.09}
\begin{itemize}[parsep=0.5ex]
  \item 2022-2023 暑期专业实习(信息工程)
  \item 从预训练的分割大模型中提取特征嵌入
  \item 图像Mask匹配算法设计
\end{itemize}

\datedsubsection{\textbf{RNA功能相关预测与RNA设计}}{2023.03 \textasciitilde~2023.10}
\begin{itemize}[parsep=0.5ex]
  \item 第43期上海交通大学本科生研究计划(PRP)
  \item 基于GloVe自然语言处理库的RNA序列和二级结构数据嵌入
  \item 基于线性CNN和GNN的结构特征提取和序列设计
\end{itemize}

\datedsubsection{\textbf{水风光储一体化多能互补计算平台}}{2022.07 \textasciitilde~2022.08}
\begin{itemize}[parsep=0.5ex]
  \item 基于SpringBoot框架的Web应用后端开发
  \item MySql数据库设计与维护
  \item API接口文档撰写
\end{itemize}

% \datedsubsection{\textbf{人体手势照片识别与分类}}{2022.05 \textasciitilde~2022.06}
% \begin{itemize}[parsep=0.5ex]
%   \item 2021-2022 Spring ICE2302-机器学习 课程设计
%   \item 简单卷积神经网络模型设计和训练
%   \item 在Spring数据库上的10折交叉验证中达到98.8\%准确率
% \end{itemize}

\datedsubsection{\textbf{基于遗传算法的矩阵特征值求解}}{2020.12 \textasciitilde~2021.01}
\begin{itemize}[parsep=0.5ex]
  \item 2020-2021 Fall MATH1205-线性代数 课程设计
  % \item 种群和状态编码设计
  % \item 针对较大矩阵和稀疏矩阵的性能优化
\end{itemize}

% \datedsubsection{\textbf{智能开关窗器设计与原型制作}}{2020.10 \textasciitilde~2020.12}
% \begin{itemize}[parsep=0.5ex]
%   \item 2020-2021 Fall ME1221-工程学导论 课程设计
%   \item 基于循环神经网络的本地天气时序数据预测
%   \item 基于深度学习的开关窗行为决策
% \end{itemize}

% \datedsubsection{\textbf{基于神经网络的字母灰度图像去噪}}{2018.10 \textasciitilde~2018.11}
% \begin{itemize}[parsep=0.5ex]
%   \item 洛谷公开赛 \#13614(原定)赛题提供
%   \item 英文字体设计与带噪图像数据库生成
%   \item 基于梯度下降算法的全连接神经网络设计与训练
% \end{itemize}
\section{荣誉奖励}

\datedsubsection{\textbf{Codeforces Contest rating:} Expert, 1744}{2023.4}

\datedsubsection{\textbf{NOIP提高组一等奖}}{2018.11}
\section{专业技能}

\begin{itemize}[parsep=0.5ex]
  \item \textbf{语言能力}
  \begin{itemize}
    \item 英语六级:569(228 + 200 + 141)
    \item 托福:104(Listening 28, Reading 28, Speaking 21, Writing 27)
    \item GRE:320
    \item 日语N1:115(31 + 52 + 32)
  \end{itemize}
  
  \item \textbf{开发环境:} Visual Studio Code,IntelliJ IDEA,Code::Blocks,Dev C++
  \item \textbf{编程语言:} 熟悉C++、Python、MATLAB,了解Java
  \item \textbf{其他:} PyTorch,React
\end{itemize}

\end{document}
